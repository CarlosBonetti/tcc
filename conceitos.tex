% !TEX root = main.tex

\chapter{Conceitos e fundamenta��o te�rica}

Neste cap�tulo ser�o feitas uma breve descri��o e conceitua��o b�sica do processo de produ��o e avalia��o cient�fica no Brasil, com o objetivo de familiarizar o leitor com a problem�tica envolvida neste processo e o que o SILQ prop�s para automatiz�-lo. Tamb�m ser�o introduzidos os conceitos tecnol�gicos e computacionais que de alguma forma foram explorados pelo sistema para chegar a este objetivo.

Esta revis�o de literatura, por�m, n�o ser� exaustiva ao ponto de revisitar os mesmos assuntos abordados no trabalho original de \cite{Silq1} (\citeyear{Silq1}) e focar� somente nos pontos onde este novo trabalho divergiu. Alguns assuntos, entretanto, foram expostos novamente com o intuito de facilitar o processo de entendimento deste artigo como um trabalho independente. � o caso, por exemplo, das se��es relacionadas aos Qualis, Lattes e das tecnologias base do sistema.

\section{Produ��o e avalia��o cient�ficas no Brasil}

\subsection{CAPES}

\afazer{TODO}

\subsection{Qualis}

\afazer{TODO}

\subsection{Lattes}

\afazer{TODO}

\section{TODO: conceitos computacionais}

\afazer{REST, API de Integra��o}

\afazer{XML - XPATH - Extra��o}
